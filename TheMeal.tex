\documentclass[12pt]{book} 
\renewcommand{\baselinestretch}{1.5} 

\newcommand{\sbreak}{
\begin{center}
  \#
\end{center}
}

\author{Bartosz Milewski}
\title{The Meal}
\date{}
\begin{document}
\maketitle{}

"There has never been a more important meeting in the history of N'jala," A'mana was told. She was painfully aware of the difficulty and the significance of her task. 

The war against the Burrions had been stewing for over three centuries before A'mana was born. Lately it wasn't going very well for the N'jala. So the recent unexpected overture from the Burrions' high command was very welcome. 

A'mana knew from history that, during the barbaric period, wars involved the killing of people. Eventually it became clear that wars were won not by the ability to annihilate the enemy, but by mastering the logistics. Thus the war between the N'jala and the Burrions was the war of logistics. And, of course, the economies behind them---there's no use of logistics without strong economy. War was an optimization problem; the mathematicians were the cavalry and the programmers were the foot soldiers. 

Also, nations used to occupy contiguous territories, except for occasional exclaves that were sources of all kinds of political troubles. This was no longer the norm and there was no need for artificial boundaries. Modern wars had nothing to do with territorial conquests.

The purpose of the meeting between the two representatives of the warring nations was to test the good will of the Burrions. So this is what A'mana was carefully preparing for. If the meeting was successful, it would be followed by a reciprocal show of the N'jala good will. 

In some cultures, food preparation people were called cooks; as if the thermal processing of ingredients were of any importance. They were also called chefs, as if the way they controlled and coordinated the process was their main task. 

The N'jala had food composers. 

But even the best food can be wasted on a consumer who can't properly appreciate it. So food consumption in N'jala was an art in itself. Eating a meal was performance art, and there were famous gourmands, who excelled in consuming the meals prepared by famous composers. There were three aspects of the art: emotional, intellectual, and physical. They were all taught in schools from the very early age. Artful consumption required a very high level of sophistication, knowledge, and years of training. 

But nurture wasn't enough; it had to be supported by nature. A person with a defective sense of taste or, more importantly smell, could never fully appreciate the food prepared by the best composers, or the selection of wines suggested by the best sommeliers. Some sectors of the N'jala society refused to participate in the genetic lottery. The ones who could afford it started directly manipulating the genes of their children. A'mana's family had the means and the will to endow her with the rare sophistication of a top performer. A'mana was a natural---by design.

When they sat at the table, the first thing that caught her attention was Ragan's jaw. There were some people in N'jala who had similar facial features, but they were considered primitive. She tried not to let this observation bias her judgment. Different cultures had different beauty standards. Burrions were, so to say, more on the bulky, muscular side of the spectrum. A'mana couldn't make herself stop comparing Ragan to the delivery guy she once knew. 

When the first dish was served, Ragan had no problems dealing with it. He used all the right utensils, picked the right pace, and made all the right sounds. Of course, all this could have been faked. Many people used an AI to quietly and undetectably guide them through the whole process. In polite N'jala society this was considered rude. So far it looked like Ragan either wasn't using any help or was very good at hiding it. He wasn't making any obvious mistakes, but he wasn't perfect. Here and there he could use some more training. But little imperfections added the air of spontaneity.

As the meal progressed, things were looking good. Then the time has come for the ultimate test. It was extremely difficult for A'mana. She had to overcome all her instincts and training. She had to do the unthinkable. It was so hard for her that she was wondering if there wasn't something engraved in her genetic code that made it mentally or even physically painful. 

In one awkward movement she pinned the writhing shamah with her douma fork. This was an unforgivable breach of etiquette. Shamah was considered moist food, and one of the basic lessons taught at every N'jala kindergarten was that you never hoist the moist. A douma fork was for hoisting. 

She knew that Ragan saw it and, if he was as well versed in the N'jala culture as he pretended to be, he'd have to be shocked. The question was, what was he going to do presently?

The N'jala culture was highly codified and, obviously, Ragan had studied it in great detail. He knew the rules. But this was something outside of the norm---there were no rules that would dictate the behavior in response to a breach like this. 

A'mana didn't look up from her plate, but she was fully aware of the shock that her action must have caused. No amount of training in N'jala lore could have prepared Ragan for this. 

The obvious thing would have been for Ragan to ignore A'mana's faux pas and pretend that he didn't notice it. But that would be deceptive. The whole meeting would be exposed as yet another Burrions' deception. There could be no peace with a dishonest adversary. 

If, on the other hand, Ragan as much as raised an eyebrow in the acknowledgment of A'mana's mistake, she would be humiliated. It would have meant that the peace plan that the Burrions had in mind was designed to humiliate the N'jala. Humiliation was worse than death.

Without any guidance or precedent, Ragan had to improvise. And he did the only thing that made sense: he took a sip of the spiced Rissa wine from his heated glass. Any kind of warm spice combined with the shamah was a major mistake, and he immediately started coughing. His face turned red. A chunk of shamah flew out of his mouth and landed in A'mana's plate. He apologized profusely, but the meal was ruined. 

At this point there was no sense in continuing the disastrous performance. The right thing to do would have been to get up and leave. Instead A'mana raised her glass, looked at Ragan, took a deep breath, and in one big gulp finished her Rissa wine. 

As far as the N'jala were concerned, the peace offering was accepted.
\end{document}
























