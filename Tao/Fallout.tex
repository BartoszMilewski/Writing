\documentclass{memoir}

\author{Bartosz Milewski}
\title{Fallout}
\date{}
\begin{document}
\maketitle{}
''I watched a movie last night,'' I said. ''And it made me think.''

''Movies often make us think,'' said the Master. ''Good movies, like life itself, ask a lot of questions, but rarely provide answers.''

''Well, that's the thing, Master,'' I said. ''Maybe you know the answer to this question, or maybe you can steer me towards the answer. I'm sure this problem has been analyzed before by many people much wiser than yours truly.

''It's a problem of moral nature. In the movie, agent Hunt faces a dilemma. His friend is in immediate mortal danger. Hunt can save him, but at the risk of endangering the lives of thousands of innocent people. He makes a choice, saves the friend but, in the process, the terrorists get hold of plutonium, which they use to make nuclear bombs. Of course, in the movie, he's ultimately able to avert the disaster, disabling the bombs literally one second before they're about to go off. 

''Sorry if I spoiled the movie for you, Master.''

''Don't worry, I've seen the movie,'' said the Master.

''So what do you think, Master? Was agent Hunt acting recklessly, risking uncountable lives to save one?''

''And what's your opinion?'' asked the Master.

''I think the answer is clear. It's just simple math: one life against thousands. I would probably feel guilty for the rest of my life for sacrificing a friend, but what right do I have to risk thousands of innocent lives?''

''You say it's simple math,'' said the Master. ''I presume there is an equation that calculates the moral value of an act, based on the number of lives saved or lost.''

''It's not an exact science, but I guess one could make some rough estimates,'' I said. ''I've read some articles that mostly deal with pulling levers to divert trolleys. So this seems like one of these problems, where your friend is tied up on one track, and thousands of people on another. A runaway trolley is going to kill your friend, and you pull the switch to divert it to the other track, possibly killing thousands of people.''

''If this is simple math, then why do you say you'd feel guilty? Shouldn't you feel satisfied, like when you solve a difficult equation?''

''I don't know. I think I would always speculate: What if? What if I saved my friend and, just like in the movie, were able to avert the disaster? I'd never know.''

''And what if you saved your friend's life and the bomb exploded?'' asked the Master.

''I guess I'd feel terrible for the rest of my life. And I would probably be the most despised person on Earth.''

''And what if that explosion prevented an even bigger disaster in the future?'' asked the Master.

''And what if that bigger disaster prevented an even bigger disaster?'' I asked. ''Where does this end? Are you saying that, since we cannot predict the results of our actions on a global scale, then there is no moral imperative?''

''Would that satisfy you?'' asked the Master.

''No, it wouldn't!''

''Would you like to have a small set of simple rules to guide all moral decisions in your life?'' asked the Master.

''When you put it this way, I'm not sure. I think there's been many attempts at rule-based ethics, and they all have exhibited some pretty disastrous failure modes. It vaguely reminds me of the Goedel's incompleteness theorem. No matter what moral axioms you choose, there will be a situation in which they fail.

''On the other hand, rejecting the axioms may lead to an even bigger tragedy, like in the case of Raskolnikov.''

''Do you see similarities between Raskolnikov and agent Hunt?'' asked the Master.

''They both reject the `Thou shalt not kill' commandment. They both feel intense loyalty to their friends and family. But Raskolnikov had a lot of time to think about his choices, he even published an article about it; whereas Hunt acted impulsively, following his gut feelings. One was rational, the other irrational.''

''But you said that Raskolnikov had no axioms,'' said the Master. ''So how could he rationally justify his actions?''

''I see your point,'' I said. ''He was trying to do the math. Solve the ethical equation. His hubris was not in rejecting the accepted axioms, but in believing that he can come up with a better theory. So, in a way, agent Hunt had the advantage of being a moral simpleton.''

''He was the uncarved wood,'' said the Master.

\end{document}








