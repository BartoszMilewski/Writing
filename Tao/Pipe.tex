\documentclass{memoir}

\author{Bartosz Milewski}
\title{The Pipe}
\date{}
\begin{document}
\maketitle{}

It was a pleasant evening and I was enjoying the warm breeze coming from the mountains bringing with it the smell of pine and something else. 

``Why are you smoking a pipe, Master? It's bad for your health.''

``This is not a pipe,'' said the Master.

``This is most definitely a pipe. You must have bought it in a pipe shop,'' I said.

``This is not a pipe,'' said the Master.

I thought for a moment.
``Oh, I see. You are making reference to the famous painting by Rene Magritte, right? Ceci n'est pas une pipe! But what Magritte meant was that it wasn't a pipe---it was a picture of a pipe. He played on our confusion between the object itself and its representation. But here you are holding an actual pipe.''

``What makes you think it's a pipe?'' asked the Master.

``Well, I can look up the definition of a pipe and you'll see that it describes the object you are holding. Do you want me to do that?'' I asked.

``Please do,'' said the Master.

I pulled out my tablet and started tapping.

``What's the wi-fi password for today, Master?'' I asked.

``That which cannot be named,'' said the Master.

``Oh, I know that one,'' I said and continued tapping. I stopped after a few tries and looked up.

``I tried Tao and Dao, upper- and lowercase, but it didn't work. Is it `the Tao,' with `the'?''

``You are much too clever, my Disciple,'' said the Master.

``Oh, you mean it's literally 'that which cannot be named'?'' I started tapping again.

``Okay, here it is. According to the OED, a pipe is...'' I hesitated for a moment.

``Wait, you don't really want me to read the definition,'' I said.

``No,'' said the Master.

After a moment of silence, I said:

``You have corrected me, Master, because by naming the pipe I focused on just one small aspect of it. Its relationship to other pipes. By doing that I ignored its relationship to you, to me, to our conversation, to this lovely sunset, to Magritte and---now I get it---to the Tao.''

``But, Master, I'm confused,'' I said after a while. ``When you say that `The Tao that can be named is not the eternal Tao,' you are giving it a name, aren't you? You are calling it the Tao.''

``This is not a name,'' said the Master.

``I have the feeling that if I say that this is indeed a name, I would be hitting a dead end,'' I said. 

We sat there for a moment while I was organizing my thoughts. Then it occurred to me.

``By calling it the Tao, you are not separating it from everything else, because the Tao is in everything. Neither are you ignoring its relationship to yourself or to me, because you are the Tao and I am the Tao. And the lovely sunset, and the pipe, and Magritte, it's all the eternal Tao.

``This is not the Tao,'' said the Master.
\end{document}
