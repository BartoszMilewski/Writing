
\documentclass{memoir}

\newcommand{\starbreak}{%
\begin{center}
  $\ast$~$\ast$~$\ast$
\end{center}
}

\author{Bartosz Milewski}
\title{The Toad and The Grasshopper Detective Agency}
\date{}
\begin{document}
\maketitle{}



Being a toad has its advantages. For instance, toads are very patient. Being patient helps in this line of work. 

Mr Toad was on a stakeout for the last twelve hours. He didn't move a muscle during that time. Frogs are patient too, but they can't stay out of water for very long, or their skin will dry out. Mr Toad had thick pimply skin. It wasn't beautiful, but it was very practical. Mr Toad was a pragmatic creature and, let's be honest, toad's esthetic sense is very different from, let's say, that of a peacock. In fact, Mr Toad considered peacocks downright ugly. What one animal sees as warts another sees as works of art and vice versa. Mr Toad could blend with the environment, which was a tremendous asset in his profession.

These kind of thoughts went through Mr Toad's head while he was on the stakeout; maybe not with that speed--he was a slow thinker, but a thinker nevertheless. You had to be a thinker in his line of work. A private eye has to be able to solve puzzles. And he was facing an unusual puzzle. 

It all started with a visit by a highly animated caterpillar. 

``I am the last of my kind,'' said the caterpillar. ``All my mates have disappeared one by one. A crime of unspeakable proportions is being perpetrated and I want you to find the killer, or killers, if there's more than one; which is quite likely considering the scale of the disaster.''

``Calm down,'' said Mr Toad and sighed. He's heard this story many times before. Every year a few panicked caterpillars would come to his office and ask for help.

``Listen,'' he said patiently. ``There is something you should know about the life as a caterpillar. Come end of summer, you all guys go through something called 'metamorphosis.' You turn into this ugly thing called a chrysalis. It sort of looks like a leathery turd, but it's not dead. Eventually you come out of it as a butterfly. These are the facts of life. There's nothing to be alarmed about.''

``No, you don't understand,'' said the caterpillar. ``This is a children's story. Nobody believes it. I am dead serious. My friends are all dead.''

``No, they're not,'' explained Mr Toad patiently. ``They turned to chrysalises. If you poke them, you'll see that they can wiggle their tails. They're still in there.''

``You're not listening,'' shouted the caterpillar. ``They're dead! They fell off the leaves and turned to mush. I don't know why I'm still alive. I guess I was lucky. But they're dead! All my friends are dead!''

``Hmmm,'' said Mr Toad. ``Let me think for a moment.

``So, you're saying that your friends are dead. You've seen their bodies and they don't look like leathery mummies?''

``No, definitely not!'' 

``Come to think of it, this is not the end of summer...''

``No, it's spring,'' said the caterpillar. 

``You're right,'' said Mr Toad.

He ended up taking the job, even though, or maybe because, the unhappy caterpillar soon started foaming at the mouth, and then promptly died in considerable agony. This is why Mr Toad was presently on the stakeout in the cabbage patch, trying to catch the grisly caterpillar killer. He was rehashing the details of the case in his head when a sudden shout ``Run!'' startled him out of his reverie. 

``Run for your life!'' he heard.

It was a grasshopper taking huge jumps, running away from some real or imaginary danger. Mr Toad followed him without much hesitation. 

'Better safe than sorry,' he thought heading towards the cabbage patch boundary. Soon a giant piece of farm machinery ran over the very spot where Mr Toad was hiding moments ago. It sprayed a fine mist of liquid over the heads of cabbages.

``That was a close call,'' said the grasshopper. 

``Indeed, my dear friend,'' said Mr Toad. ``I think I might owe you my life.''

``By the way, what were you doing there?'' asked the grasshopper.

``Trying to solve the great caterpillar-killer mystery.''

``Well, there you have it. Pesticide is your guy.''

``Indeed,'' said Mr Toad. ``Too bad my client is already dead.''

\starbreak

``I'm afraid I have bad news,'' said Mr Toad to a cow. ``Your suspicions are correct. Your husband is cheating on you.''

The cow looked at him with sad eyes.

``Do you want to see the pictures?'' asked Mr Toad, waving a thick yellow envelope in front of her. 

The cow shook her head. 

'Bulls will be bulls,' he wanted to say, but he knew it would be totally inappropriate. He felt sorry for the cow, even though he's been dealing with cases like this on a regular basis. Cows were his frequent clients, and it had always been about their husbands' infidelity. 

``Look, he's a stud. It doesn't mean he doesn't love you. They just bring one young cow after another in front of him, and he does his thing. There is no candlelight or romantic music. No cuddling or gentle kisses. It's just business. You shouldn't feel bad about it.''

``Please, spare me the details,'' said the cow.

``Tell you what, I know a great marriage counselor,'' he said. ``She's a cheetah.

``I mean, a very fast feline,'' he added, a little embarrassed. ``Here's her card.''

He pulled a card from his pocket and handed it to the cow, who looked at it with myopic eyes.

`I bet she's going to eat it the moment she leaves my office,' he thought. `Well, at least it won't go into a landfill.'

Mr Toad was a compassionate fellow. After years of practice, he was able to sympathize with mammals, even though their procreational habits were totally alien to him. But he could imagine that their emotions were comparable with what he felt when clinging to a female toad, piggyback, sometimes for days, fertilizing a long string of eggs she was excreting. Just thinking about that made his eyes fog over. He didn't see the cow leave his office. 

`I hope she didn't notice my momentary lapse of concentration,' he thought, presently recovering his customary air of professionalism. 

\starbreak


``Somebody stole my lamb,'' the lioness was visibly upset. ``I've been tenderizing it for three days, and I was ready to kill it for tonight's feast when it disappeared.''

Mr Toad felt uneasy when dealing with large predators. Not because he was afraid for his life--predators would not harm him. As food, he was considered too disgusting, if not outright poisonous; and he was marginally useful to them as a detective. 

It was a matter of trust. He just couldn't trust a large cat. A large cat never had the need to negotiate. He or she would just go for it, whatever it was that they currently needed--or whatever they fancied. And damn the consequences! In little cats the sense of entitlement was annoying, in big cats it was scary.

``Do you suspect anybody?'' asked Mr Toad.

``You bet I do. There are always these disgusting hyenas lurking around, just waiting to steal something from you. Somebody should put them in their place once and for all.''

Mr Toad didn't think hyenas would be brave enough to steal a live lamb from a lioness, especially not one with a cub. The cub was waiting patiently outside, stretched comfortably under the tree, intermittently nodding off or chasing the buzzing flies with its large paws. 

``May I speak to your cub?'' said Mr Toad.

``Why? What's wrong?'' asked the lioness.

``Oh, nothing. Sometimes kids see more things than we grownups do. They can be very perceptive. I just want to be thorough.''

``Suit yourself,'' said the lioness.

Mr Toad went outside and approached the cub.

``Hey kid, how are you doing?''

The cub didn't respond.

``Your mom tells me that the lamb you were to have for dinner is lost. Do you know anything about it?''

The cub looked up at Mr Toad and shook its head.

``Does it make you sad?'' asked Mr Toad.

``I'm not hungry,'' said the cub.

``I gather the lamb was with you for three days, is that correct?''

The cub nodded.

``You, lions, have to kill other animals to survive. That's just nature, I guess. You don't have problems killing other animals, do you?''

``No, I don't,'' said the cub.

``But that lamb wasn't killed instantly. It was kept alive for three days, wasn't it?''

The cub nodded.

``It was suffering. As your mom put it, she was tenderizing it.''

The cub winced. That was all Mr Toad needed to see. 

Most predators are incapable of empathy. They would starve to death if they were to empathize with their prey. They can't just go to a fast food joint and order a hamburger. They have to actually kill somebody, dismember them, and eat--all without feeling a trace of sympathy or remorse. Every now and then an anomaly occurs--a predator with the sense of empathy is born. Mr Toad could tell empathy when he saw it.

``Listen, kid,'' he said. ``I'm not gonna tell your mom what I think had happened. But sooner or later you'll have to deal with it. All I can say is that it sucks being different. You okay with that?''

The cub nodded.

Mr Toad went back to the office where the lioness was waiting impatiently.

``I figure that you must have done your due diligence,'' he said. ``You followed the tracks left by the lamb. It must have been bleeding all over.''

``The tracks led to the stream, and then disappeared,'' said the lioness.

``That's what I though,'' said Mr Toad. ``The most likely scenario is that the lamb was somehow able to break free. There isn't much that can be done about it.

``If I may make a small suggestion. It would be advantageous to kill the prey as soon as possible, so it doesn't have a chance to escape.''

``And eat second freshness meat...'' the lioness sneered. She wasn't going to take advice from a toad. She left the office and picked up the cub on the way back to her lair.

\starbreak


``You know what's gonna happen?'' said the grasshopper.

``Jesus!'' exclaimed Mr Toad, ``you startled me.'' He didn't see the grasshopper come to his office.

``What do you mean?'' he asked, regaining his composure.

``The kid's gonna die sooner or later, either killed by other lions, or by his own mother.''

``We don't know that for sure,'' sighed Mr Toad. ``I hope not.''

He started rearranging objects on his desk. Then he looked up and asked:

``What brings you here, my friend?''

``I was wondering if you'd be interested in hiring somebody to run your IT department.''

``My IT department?'' asked Mr Toad incredulously.

``You know,'' said the grasshopper, ``computers and stuff. Modernizing your business. Bringing it into the 21st century.''

``I thought you were working for the ants,'' said Mr Toad.

``Well, I did. Until they screwed me over,'' said the Grasshopper. 

``You want me to look into this?'' offered Mr Toad. ``I know some lawyers.''

``Don't bother,'' said the Grasshopper. ``Going against the anthill is a losing proposition. I count myself lucky to still have all my appendages intact.''

``I'm really sorry. Of course I will hire you. Not that I can offer much in terms of renumeration...''

``That's fine,'' said the Grasshopper. ``I don't need much. When can I start?''

``How about right now,'' said Mr Toad. 

\starbreak


``Are you mad at me?'' asked the puppy.

``No, why?'' said Mr Toad. ``Oh, I see... No, that's just my RTF, resting toad face. You see, the corners of my mouth naturally bend downwards, which makes me look stern, even when I'm smiling. 

``Look at this.'' He turned his eyeballs up and stuck his tongue out. ``Scary face, isn't it?''

The puppy smiled.

``So you want us to find your human family?'' said Mr Toad. ``When was the last time you saw them?''

``We were in a car, driving away from home,'' said the puppy. ``I don't like riding in a car. It makes me sick. I puked, and they stopped the car and let me out. Next thing I knew, the car was gone. They must have gotten confused. I think they thought I was back in the car, and they kept driving. And then I made a mistake. I panicked and, instead of waiting for them, I started running along the highway. I'm sure they turned back as soon as they noticed that I was gone, but we missed each other. They probably waited for me to come back, but I was running and running. When I finally turned around, and went back to the place they let me out, there was no-one there. They must have given up waiting for me. So I waited for them. I was so tired, I fell asleep, but I kept waking up. I was thirsty, but I didn't want to look for water, in case they came back. Fortunately, it started raining, and I was able to drink from a puddle. I was all wet and cold. I don't know how many days I waited before I gave up.''

``What do you want us do do?'' asked Mr Toad.

``You have the technology, you can look them up,'' said the puppy. ``They must have posted messages on the Internet. They are probably desperate, worried sick that something bad happened to me. I want them to know I'm okay. Their kid loved me so much, he's probably crazy with grief.''

``They have a kid?'' asked Mr Toad. ``Was the kid in the car with you?''

``No, he wasn't.''

There was a long moment of silence. 

``Oh no!'' exclaimed Mr Grasshopper from behind his computer. ``I'm so sorry! It looks like they are all dead.''

``Are you sure? What happened? What about the kid?'' the puppy kept asking questions.

``It looks like they have sacrificed their lives for you. There were some very bad people after them. Some kind of drug lords, or foreign spies.'' Mr Grasshopper was reading from the screen. ``They drove you out to save your life, and then they went back to their place, and got killed. The kid too, I'm afraid.''

``Oh no!'' the puppy was whining. Then it squatted and peed on the floor. 

``It's okay,'' said Mr Grasshopper. ``They must have loved you very much.''

The puppy ran out howling desperately.

``What was that about?!'' asked Mr Toad.

``What was I supposed to tell him?'' said Mr Grasshopper. ``The truth? And break his little heart?''

``But drug lords? Foreign spies? He's gonna figure it out sooner or later.''

``Eventually,'' said Mr Grasshopper. ``But by that time he'll be stronger and more cynical.''

``Sometimes you're scaring me,'' mumbled Mr Toad.


\end{document}

