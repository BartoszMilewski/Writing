\documentclass[12pt]{book} 
\renewcommand{\baselinestretch}{1.5} 

\newcommand{\sbreak}{
\begin{center}
  \#
\end{center}
}

\author{Bartosz Milewski}
\title{Juxtaposition}
\date{}
\begin{document}
\maketitle{}


``I have a huge problem,'' said Chuck. ``With this new mission, and the hazard pay we're getting, we decided to remodel the house. My wife has already picked color schemes for the living room and the kitchen. She wants me to decide on the bedroom. She hinted that she'd like to paint it yellow, but, I don't know, yellow is sort of like the alert color, don't you think?, and in the bedroom you want to relax, right? So I was thinking blue. What do you think?

``Indeed,'' said Sally, ``it's a tough call. I see your point about relaxing. On the other hand, when you wake up, you want to be alert.''

``That's true, I haven't thought of that.''

``Or,'' said Sally, ``you could go for a compromise.''

``What do you mean?''

``If you mix yellow and blue, you get green. You could paint the bedroom green.''

``I haven't thought of that, either,'' said Chuck. He took a long look at the display in front of him. It wasn't clear if he was studying the screen or considering color options for the bedroom. Finally he said, ``I brought some color swatches, if you want to look at them.''

``You're kidding me, right?'' Sally looked at Chuck with disbelief. ``You brought color swatches to a space mission--possibly the most important mission in the history of humankind--so that you can pick the paint for your bathroom...?''

``Bedroom.'' Chuck corrected her. ``Well, you know,'' he said, ``I thought we'd have a lot of downtime before we get there, so...''

Sally rolled her eyes, ``Okay, show me.''

``Wait, what's that?'' Chuck was looking at the screen. The previously static image of the artifact in the center was undergoing a fast transformation. 

``Are you seeing what I'm seeing?'' he asked.

``It's like some kind of metamorphosis,'' said Sally, ``A butterfly emerging from a chrysalis?''

``It's full of colors,'' said Chuck. 

\sbreak

``Is this really necessary?'' asked John. The assistant was applying makeup to his face in preparation for a televised press conference. 

``The cameras are very unforgiving,'' said the assistant. ``Without the makeup your face will look all blotchy and sweaty. The tiniest blemishes will be magnified.''

John sighed in resignation.

``Also, performing in front of the public might be a little intimidating,'' she said. ``That's why we would like you to concentrate on one particular person in the first row. We call him 'the mushroom guy.' He's a short stocky guy with a clean shaven head. Just imagine that you're talking to him. Apparently he has a very calming influence on speakers.''

``Fine,'' said John. ``Although I don't think this should be necessary. I've spoken to large audiences at many conferences before, and stage fright is not a problem for me.''

``Well, just in case...'' said the assistant putting final brushstrokes of powder on his face.

It was time for John to get out in front of the press corps. He stood behind the dais, half-blinded by the lights. He noticed the mushroom guy in the front row. Indeed, he had some hard to describe fungal quality. Maybe it was his shaved head that looked like a fruiting body of a fungus. He couldn't tell.

``My name is Dr John Peleas,'' he started. A few flashes from reporters' cameras blinded him momentarily. He spelled his name for the press. He realized it would be in the news almost immediately, so they better get it right.

``As you might have heard through the rumor mill, we have some strong evidence that the artifact that came on a hyperbolic trajectory to our Solar System might be of alien origin. In fact, we suspect that it might be the same object that we've discovered in 2017 and given the name 'Oumuamua. We are not completely sure. However it seems to have more or less the same shape and spin. Of course, 'Oumuamua had the escape velocity when it left the Solar System, so it could not have come back on its own. It would have to perform a change of orbit by some kind of powered flight. Our theoretical models show that this kind of maneuver would require capabilities that are way beyond our own current space technology.

``We didn't have the opportunity to study 'Oumuamua's first approach in much detail. By measuring the changes in its albedo, that is the way it reflected sunlight, we figured that is was a thin elongated shape rotating with a period of about eight hours. This time, we have much better data. The artifact's shape is quite unusual in its regularity. Moreover, the way it rotates is also unusual. Let me explain...''

John pulled out the cellphone from his pocket and showed it to the audience.

``Imagine that this is the artifact. It has three natural axes of rotation. I can throw it like a Frisbee, so it rotates along the axis perpendicular to the phone.''

He made a gesture as if throwing the phone.

``Or I can make it pirouette like a dancer.

``There's also a third way--I could make it tumble like this...''

He demonstrated the tumbling motion.

``The thing is, the tumbling movement is unstable. The slightest perturbation will be amplified and eventually will lead to a change of the rotational axis. 

``For obvious reasons I'm not going to demonstrate it with my phone, but you are free to do the experiments yourselves. Maybe with something more sturdy.

``The most unusual thing that made us take a second look at the artifact is its trajectory. As I mentioned before, it came to us at hyperbolic speed, which makes it an interstellar object. On its way towards the Sun, though, it passed very close to Jupiter. In fact it performed what we call the reverse slingshot maneuver, a hairpin turn, which slowed it down to orbital speed. Not only that-- when we calculated its future trajectory, we realized that it will make another fly-by around Saturn. This time it will go through a forward slingshot that will restore its escape velocity and put it on a path out of the Solar System.

``We've done similar caroming in the past with our Pioneer and Voyager probes and, more recently, with New Horizons. But these were all very well planned precision maneuvers. The chances of a random interstellar object doing this twice are, let's just say, astronomical. 

``As you know, we have sent a crewed spaceship to explore the artifact. This would have been absolutely impossible if the artifact was on a hyperbolic trajectory. We don't have this kind of technology. But the slowdown gave us the chance to catch up with it. It's almost like an invitation to visit.

``Sally McDermott is the commander of the ship and she's accompanied by Charles Atkey. They are currently within visual range of the artifact. We are in contact with them, and we'll keep you informed about the developments. Due to the distance, there is currently a 15 minutes lag in communications.

``I will now answer your questions.''

Several reporters started talking all at once, so John pointed at the closest person.

``Sir, do you think the aliens may be flying the artifact?'' she asked. ``Are we prepared to initiate contact with them?''

``It's a possibility,'' said John, ``although it's much more likely that it's an automated probe. At least that's what we would...''

``Excuse me,'' John was interrupted by a staffer emerging from the side entrance. He approached John and muttered something into his ear.

``I'm sorry,'' said John to the audience. ``There were some sudden changes in the artifact that we would like to analyze. We'll be back with more news soon...''

He was quickly escorted out of the conference room.

\sbreak

``I'm going to switch to infrared,'' said Chuck. ``We'll see if they have any coffee brewing. Or engines, or reactors... Or whatever their source of energy is.''

``Strange,'' said Sally. ``It's as multicolored in infrared as it is in visual. It's as if they wanted it to look interesting across the spectrum.''

``I'm switching to UV,'' said Chuck. 

``I'll be damned. Same thing there, too. How do they do it? I'm beginning to suspect that their eyes may be much better than ours. Nobody paints their ship in infrared and UV unless they can see it.''

``Let's do a spectral sweep,'' Sally said. ``We'll see if there are any peaks that we can match with standard materials.

``Nothing! The Fourier transform is pretty much continuous. It's definitely not black-body radiation, though. It has undulating hills and valleys, but no sharp lines.''

``Should we poke them with lasers?'' asked Chuck.

``And risk them poking us back? I don't think so.''

``Do you think they're there? The actual aliens?''

``You mean like the scary alien xenomorphs from the movies? I don't think so. They say this is most likely an automated probe.

``Just think of it,'' said Chuck, ``if they are capable of interstellar travel, their automated probe could very well be smarter than us.''

``You might be right,'' said Sally. ``I went on a date once, with this very handsome guy I met through a dating service. He wrote in his profile that he was a bass player, but I think it was more of a wish than reality. 

``I very quickly realized that he was, how should I put it, a low information person. 

``You know, there are these questions they ask people who have just come out of a coma. Like, what year it is, or who the current president is...

``I'm not sure he'd be able to answer all of them.

``So, obviously, we didn't have much in common, and the date was a complete waste of time.

``Which makes me wonder, are we on a date like this, except that we're the stupid ones?''

\sbreak

It was alien, but in a very unthreatening way. Chuck was on an EVA, trying to figure out the topography of the artifact, which, by now, has bloomed into a large multi-colored structure resembling a flower, a butterfly, an anemone--he couldn't make up his mind. It was something that had no analog in human experience.

A spaceship is supposed to be an enclosed space, but there was no enclosure to be found. There was no inside or outside. There was no engine compartment, living quarters, or storage tanks. In fact, it was easier to say what it wasn't than to describe what it was.

``Any luck?'' asked Sally over the radio. She stayed behind in the cockpit, watching him from a distance. ``Anything you can pry open?''

``Nothing I could use my crowbar on,'' said Chuck. ``Things are kind of fuzzy down here. Not very well defined.''

Chuck navigated deeper between the giant petals of the artifact, hoping to find the backbone of the structure, but there wasn't any. In the corner of his eye he noticed some movement and he turned to face it. It was a small shiny spot jiggling around. When he looked at it, it moved to the side. He followed it. The next moment a large bright shape obscured his view. He instinctively raised one hand to protect his face. He pushed against the shape, and it floated away while turning around. It was Sally in her white spacesuit.

``Jeez! You scared me shitless!`` he said. ``How the heck did you get here?''

``Calm down,'' said Sally. ``Do you know where you are?''

``Of course,'' said Chuck. ``Are you going to ask me what year it is?''

``I might as well,'' she said. ``You were out of it for, like, forty minutes``

``What do you mean? I wasn't! I was just following this light, whey you suddenly dropped right in front of my eyes. Did they teleport you?''

``What are you talking about?'' said Sally. ``I lost you forty minutes ago and I came to rescue you--or bring your body back, for all I could tell. I had no idea what happened to you.''

``You're serious, aren't you?'' Chuck was visibly confused. 

``You bet your ass, I am,'' said Sally. ``Let's go back and talk to Mission Control``

\sbreak

John Peleas was looking at Sally and Chuck through the glass window that formed one whole wall of the room they called `the aquarium.' The two were spending forty days in isolation. The number forty had no particular significance, other than being the direct translation of the Italian word `quarantina.' If they were infected by some alien life form, nobody had any idea how long it would take for the symptoms to show up, or what kind of symptoms to expect. `Hopefully not a tiny head with sharp teeth exploding through the ribcage,' though John.

After the EVA, and still in the airlock, the pair thoroughly decontaminated their spacesuits,  took them off, and ejected them into space. But nobody knew what kind of decontaminants worked on alien microbes, or if the spacesuits formed an effective barrier. John's imagination went through many scenarios in which tiny spores could drill their way through the fourteen layers of synthetic fabrics and infect the astronauts. But if that were true, the spores could as well be already drilling through the plastic seals framing the window of the aquarium. 

``How are you guys doing?'' he asked through the intercom. 

``Still a bit weak from two weeks in zero g,'' said Sally, ``but otherwise in good spirits.''

``As far as first contacts go,'' said Chuck, ``this one kind of fizzled out, don't you think, Dr Peleas?

``Please, call me John,'' said Dr Peleas, ``now that we can finally meet in person.''

They have already had several conversations when the space probe was in flight on its way back to Earth. 

``We went out to meet the aliens and all we got was these lousy t-shirts,'' Chuck pinched the fabric of his NASA t-shirt.''

``Before the rest of the team joins us,'' said John, ``I just wanted to say how much we appreciate your bravery.'' 

``It's our job,'' said Sally.

\sbreak

After the introductions and a bit of smalltalk, the neurologist asked the first question:
``I know we've talked about this many times, but can you tell me exactly what your perception of the passage of time was during the forty minutes of your syncope. Did it seem like you lost some time? Did you have any hallucinations?''

``I had no idea I passed out,'' said Chuck. ``I didn't feel any discontinuity. I was just startled when I suddenly saw Sally.'' 

``Did you feel any weakness or dizziness before that?''

``Absolutely nothing. It was like a jump cut. You know, like in the movies.''

``Normally a syncope is caused by a sudden drop in blood flow to the brain. But that's usually preceded by other symptoms. We have gone through the recordings of your pulse, blood oxygen levels, breathing rhythm, and so on, and we haven't found any abnormalities before, or during the episode. There was a sudden jump in the readings after you woke up, but that was probably caused by the shock of seeing Sally.

``You said, his eyes were closed,'' he turned to Sally.

``That's correct,'' she said. ``He looked like he was asleep. Or in the middle of a very long blink.''

``Without an MRI scan I can't eliminate natural causes,'' the neurologist turned to his colleagues, ``but considering the length of the episode, this was highly unusual.''

``I know how weird this sounds,'' said John, ``but considering that this happened on an alien spaceship, we have to think of the possibility of... I don't know... a brain scan done by the aliens?''

``Or an attempt at communication,'' said the biologist. 

``Failed attempt?'' said the physicist.

\sbreak-

``Let me buy you a coffee,'' said John. He and the physicist were leaving the building after another day of brainstorming with the team. They went to a nearby cafe and sat down to drink their cappuccinos.

``Did you see Chuck nodding off during our discussion?'','' said the physicist.

``Can't blame him,'' said John. ``We've been going over the same things over and over for the last two weeks without much progress.''

``They might have tried to communicate with him,'' said the physicist. ``And failed.''

``Are you saying that Chuck was not the humanity's optimal first-contact person?'' asked John.

``I don't know. He's just an astronaut--an engineer by training. Not that there's anything wrong with engineers. And, as we have just learned from the recordings, he's the guy who brought color swatches on his trip to space.

``Sally, on the other hand, has a Ph.D. in astrophysics,'' he added.

``So does Brian May,'' said John.

``Who's that?''

``Never mind... The truth is, we don't even know if they tried to communicate, or if they probed his brain, or what. Most likely it wouldn't have made any difference if they tried to talk to, let's say me, you, or Albert Einstein.''

``You're probably right,'' sighed the physicist.

``He's a nice guy, by the way,'' he added. ``And so is his family.'' 

Chuck's wife, Suzy, has been visiting him at least twice a week, bringing along his 4-year old son on a few occasions. They had to talk through the intercom. They could only touch hands through the thick glass of the aquarium. 

Sally's mother has visited her a few times. They didn't talk much but it seemed like they had a good connection. 

\sbreak

``We are stuck,'' said John. ``We have encountered an alien artifact and we don't even know if we've made contact. I mean, has there been any meaningful exchange of information that would deserve to be called `contact'?''

``Maybe `contact' is too big a word,'' said the physicist. ``Maybe we should call it `juxtaposition'. Two--for the lack of a better word, civilizations-- that were alien to each other have momentarily overlapped in space and time. We'd have a better chance of exchanging information with octopuses or trees...''

``Or fungi,'' interjected the biologist, dr. Pilz. ``I've been thinking about fungi, as an exercise in probing out-of-the-box ideas.

``Fungi are really strange. They defy our attempts at categorization. They grow like plants, but they don't photosynthesize. Instead of cellulose, they produce chitin, like insects or crustaceans. We can't even decide if they are single-cell or multicellular. Their mycelia are sort of divided into cells, but not really. A fungal cell grows by elongation. When it reaches a certain length, it produces a partial wall, called a septum. Those septa have holes in them and they let the organelles and nuclei move through. It's like they don't give a damn about fitting into our neatly designed categories. 

``So let's try to imagine for a moment a civilization of sentient fungi. How would it be different from ours?

``The thing that drives evolution is the preservation of life. Animals, with few exceptions, evolved to protect the individual. That makes them--us--selfish compared to plants or fungi. We value our life more than the lives of others. A mushroom may kill you, but only after you eat it. Fungi are selfless. 

``Just like modern armies, fungi use force dispersal as their defense strategy. An enemy may kill a few mushrooms, but the system survives. Dispersal requires a good communication network: the mycelium. Survivability rather than aggression is their main strategy.''

``But, as you said,'' chimed in the physicist, ``humanity as a whole employs the same survival strategies. We disperse and we accept some losses in order to survive.''

``Yes, we could compare a fungus to humanity as a whole,'' said the biologist. ``Except that humanity can be decomposed into individuals, and with fungi it's hard even to decompose them into cells. Not to mention that we often can't even tell where one fungus ends and another begins. For instance, we see mushrooms as individual entities, but they are just small eruptions on top of a large network of mycelia. In fact it's often hard to tell where a fungus ends and where its symbiont begins.

``And then there is lichen, which we consider a composite of a fungi and algae, but we can't really pry them apart.''

There was a moment of silence, which was interrupted by Chuck clearing his throat.

``May I say something?'' he said through the microphone.

``Of course. Go ahead,'' said John. Some of the scientists looked surprised at Chuck. He usually just answered questions. This was probably the first time he took the initiative.

``I've been thinking,'' said Chuck, ``about what dr. Pilz had said. How we always try to pigeonhole everything. Put it in labeled drawers. How we take things apart in order to understand them. We do it automatically, without giving it much thought.

``But what if there was a life form that doesn't do it? Doesn't need to decompose things. Sort of takes things holistically.

``When we were approaching the artifact, we were thwarted in our attempts to categorize it. We tried to do spectral analysis, attempting to decompose its emanations into discrete wavelengths. That didn't work. We tried to identify its components: an engine, a navigation system, living quarters. We failed. We couldn't even figure out if there was an inside and the outside of this thing.

``That's very different from how we design our spaceships. Things are always compartmentalized, each part serving some unique purpose. We build things from little specialized blocks. 

``We are multicellular organisms inside and out. Our thinking is cellular. 

``I'm trying to imagine what it would be like not to try to decompose everything and, frankly, I can't. 

``So how would we communicate if one side could only understand things by chopping them into small pieces, and the other would see this chopping as destruction?

``When we break things apart they stop working. And we have no idea how to understand the whole thing at once without splitting it up.

Chuck stopped and there was a moment of silence. Encouraged, he continued.

``Life itself, the way we understand it, requires that we distinguish between the living being and the environment. Physically this was first accomplished by creating a barrier, a cell wall that decomposed the world into the inside and the outside.'' He touched the glass wall separating the astronauts from the scientists. ``This wall here is to protect you from us. We might be carrying some lethal microbes or parasites, and you want to keep us quarantined.

``So the natural assumption is that the environment is hostile. There is the evil on one side and the good on the other. 

``How do we protect ourselves from the evil? We have to be able to anticipate its actions. 

``You put us here, in quarantine, because you were anticipating the possibility of contagion. You constructed a model in your mind, and when you ran this model, one of the scenarios was that we would pass some kind of disease to you; a disease that could spread to the rest of humanity and wipe it out.

``Every living being we know of, whether it's a bacterium or a human being, must create some model of its environment. The models we humans create are very sophisticated. Those used by bacteria are comparatively trivial. But even bacteria react to environmental stimuli. This could be purely mechanical, or it could be the behavior encoded in their DNA. Show me your DNA and I'll tell you what your environment is like. A model of the environment is encoded in the genetic material.

``And a model is necessarily composed of parts: be it amino acids in the DNA, or synaptic connections in our brains. Since the model is decomposable, we assume that reality is decomposable as well. But that's just our hubris speaking. I'm not a physicist, but we can ask John's opinion: Does physics tell us that the Universe is decomposable?''

John felt a little put on the spot by this direct question. He hesitated for a moment.

``Well, all our theories are obviously trying to decompose the Universe into small digestible pieces. We use math, which can be defined as the science of decomposition, and we've been very successful at that so far.

``But, as you say, we are only building models of reality. Does it mean reality is decomposable? I don't feel competent to answer this obviously philosophical question. I'd like it to be, otherwise I'd have to acknowledge the limitations of not only physics, but the whole scientific method.

``But as long as we don't have the ultimate Theory of Everything, all bets are off. We do have the Standard Model, and we have General Relativity, but the two seem to be incompatible with each other. So there's that. And the Standard Model covers less than five percent of the Universe. The rest is dark matter and dark energy, about which we can only speculate. I wish I had better news, but this is the honest truth.''

``So is it even possible for any form of life to forego decomposition?'' said the biologist. He looked at Chuck. ``Aren't you asking us to think of something that, by definition, is impossible to think about for us, human beings? A leap of faith?''

``I wish I knew,'' said Chuck. ``I'm as confused as you are. But that's the gut feeling I get from seeing and-- I don't know, interacting?-- with the artifact.''


\sbreak

The Atkey family greeted the end of the quarantine, and Chuck's return home, with a big barbecue. After everybody finished their burgers and ate the desserts, Chuck helped with the dishes and tucked the kids in bed. He then joined Suzy in the living room to sip a glass of port. 

``Okay, Chuck,'' she said, ``now you can tell me everything. What secrets did the aliens pass to you?''

``Well, they told me about the meaning of life.''

``What did they tell you?''

``Family! The most important thing in life is family. That's what they told me. They said: `Chuck, cherish your wife and your kids because there's nothing more important in life than the family.'``

``Aw shucks, you're such an incurable romantic.'' Suzy put her arm around him and kissed him on the cheek.

``What, you don't believe me?'' Chuck said laughing.

``Tell me,'' she said, ``did you just lose consciousness for forty minutes? Or was it some kind of amnesia? Maybe something will trigger your memory.''

``Maybe. I don't know,'' 

Chuck got quiet for a moment. 

``There is one thing, though,'' he said. He reached into his pocket and pulled out the color swatches. ``After all this strange experience it became absolutely clear to me what color we should paint the bedroom.'' 

He pulled one swatch and pointed to it. ``That's the one.''

``You're kidding me,'' said Suzy.

``Actually, I'm not,'' said Chuck.

\sbreak

Sally looked at the clock. It was already past ten and it was time for her mother to get the herbal tea and go to bed. It was part of the nightly ritual.

Sally was worried about her mother's health. She's been forgetting a lot of things lately. She would start a sentence and stop in the middle, trying to remember the right word. She'd use generic descriptions, like `the thing you use when it rains,' when she meant an umbrella. She was aware that her memory was deteriorating. She started looking for an assisted living facility for herself. `I don't want you to see me like that,' she said. She's always been a proud and independent woman. 

Sally filled two cups with water and put them in the microwave. She was making one for herself too. There was no reason to stay up late, and the herbal tea calmed her nerves.

The microwave dinged twice and she took out the steaming cups using a dish towel to protect her hands. The clock on the microwave showed ten fifteen. She dunked the two sachets of chamomile tea in the cups to let them steep.

``Sally, what's taking you so long?'' It was the mother calling from the bedroom. 

Sally took the cups. Strangely, they were not hot at all. Barely warm in fact. She glanced at the clock: it was five minutes to eleven. She took the tea to her mother's bedroom.

\end{document}
























